\documentclass[a4paper, 12pt]{article}
\usepackage{booktabs}

\begin{document}

\vspace{0.5cm}
% \noident\textbf{RESUMO.}
% Neste artigo analisaremos o problema de
% transferência de calor em regime permanente em superfícies estendidas 
% unidimensionais. Começamos fazendo uma introdução do problema, em seguida apresentamos a formulação numérica do problema, resultados e conclusão. 

\section{INTRODUÇÃO:}

\section{O PROBLEMA E A FORMULAÇÃO NUMÉRICA}

O problema de transferência de calor é \ldots\newline
A abordagem numérica utilizada para este trabalho será apoiada na utilização de métodos de diferenças finitas, com ênfase em uma abordagem por diferenças centradas de três pontos. Tal formulação tem
origem na definição de derivada. A derivada é fornecida pela definição de limite \newline

\begin{equation}
  \frac{df}{dx} = \lim_{\Delta x \to 0} \frac{f(x + \Delta x) - f(x)}{\Delta x}
\end{equation}

Seja portato, $g(x) =  \dot{f}(x)$ a derivada de primeira ordem da função genérica $f$, a derivada de segunda ordem passa a poder ser escrita como uma 
% 
% \section{RESULTADOS NUMÉRICOS:}
% 
% \section{CONCLUSÃO:}
% 
% \section{REFERÊNCIAS:}

\end{document}
